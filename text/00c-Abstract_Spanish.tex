\thispagestyle{plain}
\phantom{\tiny test}

\noindent \Huge{\textbf{Resumen}}
\begin{center}
\rule{\linewidth}{.4pt}
\end{center}

\normalsize
\vspace{0.7cm}

En esta tesis se presentan b\'usquedas de nuevos fen\'omenos involucrando la presencia de quarks top y bosones de Higgs en colisiones prot\'on-prot\'on en el Gran Colisionador de Hadrones del CERN.
La primera b\'usqueda tiene como objetivo una serie de se\~nales, incluyendo la producci\'on en pares de quarks vector-like ($T$) con una fracci\'on de desintegraci\'on dominante a un quark top junto con un bos\'on de Higgs del Modelo Est\'andar o un bos\'on Z; la producci\'on de cuatro quarks top, dentro del Modelo Est\'andar y en varios escenarios de nueva f\'isica; y  bosones de Higgs pesados producidos en asociaci\'on con, y desintegr\'andose en, quarks de la tercera generaci\'on.
La segunda b\'usqueda est\'a centrada en la producci\'on del bos\'on de Higgs del Modelo Est\'andar en asociaci\'on con una pareja de quarks top-antitop, 
$t\bar{t}H$, con $H\to b\bar{b}$, con el objetivo de medir directamente el acoplamiento Yukawa entre el quark top y el bos\'on de Higgs. \par
Las b\'usquedas est\'an basadas en 13.2 fb$^{-1}$ de datos de colisiones prot\'on-prot\'on a una energ\'ia del centro de masas de $\sqrt{s}=13$$\tev$ tomados con el detector ATLAS. Para ambas b\'usquedas, los datos son analizados considerando el estado final denominado lepton+jets, caracterizado por la presencia de un
electr\'on o mu\'on aislado en el detector y con alto momento transverso, alto momento transverso faltante y m\'ultiples jets. En el caso de la primera b\'usqueda, los datos son analizados tambi\'en en el estado final denominado jets+$\MET$, que no hab\'ia sido considerado previamente en el Run 1, y que est\'a caracterizado por la presencia de m\'ultiples jets y alto momento transverso faltante. Ambas b\'usquedas explotan las altas multiplicidades de jets y $b$-jets caracter\'isticas de los sucesos de se\~nal. En la primera b\'usqueda, el alto valor de la suma escalar del momento transverso de todos los objetos f\'isicos en el estado final, y la presencia de resonancias con alto {\sl boost} que se desintegran hadr\'onicamente y son reconstruidas como jets con un valor de radio grande, son usadas para discriminar entre los sucesos de se\~nal y los sucesos de fondo, en particular aquellos provenientes del proceso $\ttbar+\ge1b$, el cual ha sido objeto de estudios detallados.\par
No habiendo encontrado un exceso significativo sobre la predicci\'on del Modelo Est\'andar, en el caso de la primera b\'usqueda se han derivado l\'imites superiores en la producci\'on de se\~nal a un nivel de confianza del 95\% para una serie de modelos de referencia, en la mayor\'ia de los casos mejorando de manera significativa la sensitividad de b\'usquedas anteriores. En el caso de la segunda b\'usqueda, los datos se muestran consistentes tanto con la hip\'otesis que excluye la presencia del proceso $t\bar{t}H$, como con la hip\'otesis que lo incluye de acuerdo a la predicci\'on del Modelo Est\'andar. La raz\'on entre la secci\'on eficaz de $t\bar{t}H$ medida y la correspondiente predicci\'on del Modelo Est\'andar es $\mu=2.1^{+1.0}_{-0.9}$, asumiendo un bos\'on de Higgs con una masa de 125\gev.

\vspace{0.5cm}

\begin{center}
\rule{\linewidth}{.4pt} 
\end{center}
\textbf{Palabras clave:} f\'isica de part\'iculas, CERN, LHC, ATLAS, quark top, bos\'on de Higgs, modelo est\'andar, b\'usqueda, nuevos fen\'omenos, Monte Carlo.
\begin{center}
\rule{\linewidth}{.4pt}
\end{center}




