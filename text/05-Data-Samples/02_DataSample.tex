\section{Data sample}
\label{chp:sec:data}

The analyses presented in this dissertation are based on $pp$ collision data at $\sqrt{s}= 13$ $\tev$ collected 
with the ATLAS detector between August and November 2015, and between April and July 2016. A total integrated luminosity of 13.2 fb$^{-1}$
was recorded under stable beam conditions and requiring all subdetectors to be fully operational during data taking. Events are selected using the lowest unprescaled single-lepton or \MET triggers.\par
Single-lepton triggers with low $\pt$ thresholds and lepton isolation requirements are then combined in a logical OR with higher-threshold triggers without isolation requirements in order to increase the overall efficiency. Different thresholds were used in 2015 and 2016 to keep the output rate $\sim1.5$ kHz despite the increase of the instantaneous luminosity. For muon triggers, the lowest-$\pt$ threshold is 20 (24) GeV in 2015 (2016), while the higher-$\pt$ threshold is 40 (50) $\gev$. For electrons, isolated triggers with a $\pt$ threshold of 24 $\gev$ are used with non-isolated triggers at 60 $\gev$ in both years, along with trigger with a threshold at 120 (140) $\gev$ requiring looser identification criteria. 
The isolation requirement that is applied offline is tighter than the one included in the trigger; therefore, the analyses are not affected by the isolation requirement applied at the trigger level.

The \MET trigger \cite{ATL-DAQ-PUB-2016-001} considered uses an \MET threshold at the HLT level of 70 (100) $\gev$ in 2015 (2016) and becomes fully efficient for an offline $\MET>200$ \gev.

