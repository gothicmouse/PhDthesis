
This chapter presents the feasibility of a search at the LHC for a light ($m_{A}\le100$ \gev) CP-odd scalar produced with a top-quark pair and its subsequent decay into a $b\bar{b}$ pair.
CP-odd scalars are introduced in several SM extensions, e.g. Supersymmetry or a general N  Higgs Doublet Model, and they are natural candidates for dark matter. For example, in the context of Coy Dark Matter models \cite{Arina:2014yna,Hektor:2014kga,Boehm:2014hva}, CP-odd scalars are used as mediators between dark matter and SM particles to explain the diffuse gamma-ray excess from the Galactic Centre \cite{Goodenough:2009gk,Hooper:2010mq,Abazajian:2012pn,Daylan:2014rsa}. Their elusive nature because of the absence of direct coupling to gauge bosons,  reflects in weak constraints from direct searches \cite{Craig:2015jba,Hajer:2015gka,Dolan:2014ska} on the mass of CP-odd scalars, so that light scalars with $m_{A}\ge5$ $\gev$ are still allowed. This search targets the lepton+jets $t\bar{t}$ final state. A brief description of the simplified-model approach used to generate the signal samples is presented, as well as the simulated and the event reconstruction. A novel analysis strategy to target this final state, the results obtained and their interpretation in the context of 2HDM and NMSSM models is also discussed. This study was published in \cite{Casolino:2015cza}.          
