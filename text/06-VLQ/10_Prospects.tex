\section{Future improvements}
\label{chp:VLQ:sec:improvemente}
Although this search has shown excellent sensitivity over a wide range of BSM signals sharing the same signature, further improvements are still possible. They can be divided in different categories:

\bi
\ib Event categorisation: currently events are categorised using the multiplicity of mass-tagged jets, which efficiently tag boosted hadronic Higgs bosons and top quarks but without any further identification. An improved categorisation using the multiplicity of boosted hadronic top-quarks and the multiplicity of boosted Higgs bosons present in the event can improve the discrimination between $T\bar{T}$ production and $t\bar{t}+$jets background using the presence of multiple boosted particles in the $T$-quark decay. A simple implementation of this event categorisation can be done using a mass window on the RT-jets mass instead of a lower cut as present in this analysis. Preliminary studies using this refined categorisation lead to an improvement on the mass reach of $\sim50$ $\gev$. A splitting of the event categories using number of jets can improve the sensitivity to four-top-quark scenarios since those signals live at very high jet multiplicity ($\ge8$ jets). 
\ib Extension of the 0-lepton analysis: the design of the current 0-lepton analysis leads to high sensitivity for the $T\bar{T} \to HtZt$ decay, covering a wide range of possible branching ratios, provided the ${\rm BR}(T\to Wb)$ is not too high. A possible improvement would be to extend the sensitivity to higher ${\rm BR}(T\to Wb)$ at high mass, targeting the $T\bar{T} \to ZtWb$ decay.
This can be explored allowing for lower jet and $b$-tag multiplicities and targeting the decay of boosted hadronically-decaying $W$ bosons whose decay products would be collimated in a small-$R$ jet for high $m_{T}$ masses (>1 $\tev$).  
\ib Extension of the 1-lepton analysis: the current 1-lepton analysis is very powerful for high values of ${\rm BR}(T\to Ht)$. It is possible to extend the reach making this analysis more sensitive to high ${\rm BR}(T\to Zt)$ exploiting the presence of a $Z \to \nu \bar{\nu}$ decay. This can be achieved by adding a 1-lepton high-\MET channel selected via the \MET trigger, and splitting the analysis in low-\MET and high-\MET regions. In the case of the high-\MET channel, further splittings of regions according to powerful variables such as $m_{T,{\rm min}}^{b}$ (used in the 0-lepton channel) should lead to improved sensitivity.
\ib Discriminating variable: the current discriminant $m_{\rm eff}$ has good discrimination for hard signals ($T$ quark, $t\bar{t}t\bar{t}$ in EFT and UEDRPP model) but it's not optimal for softer signals (SM $t\bar{t}t\bar{t}$ and $b\bar{b}H/t\bar{t}H/H^{+}tb$ production). The use of multivariate techniques for those signals benchmarks would lead to an improved sensitivity.
\ib $t\bar{t}$ modelling: ultimately this search is limited by large uncertainties that affect the main background. Further refinements on the background prediction through NLO MC simulations matched to parton shower (MEPS@NLO) will be important to obtain the most accurate possible modelling.
\ei
