
The ATLAS detector records events as raw data, which correspond to bits of electric signal collected when particles interact with the detectors. The output information from all subdetectors is combined to form basic quantities such as tracks and calorimeter clusters. Finally, a particle identification step is performed, where the information from the relevant subdetectors is combined to reconstruct as accurately as possible a candidate physics object. In this chapter the techniques used to reconstruct, identify and calibrate the physics objects used in the analyses presented in this dissertation are described. These objects include: tracks and vertices, electrons, muons, jets, and missing transverse momentum. A brief description of the systematic uncertainties associated with these physics objects is also provided.
