\chapter*{Introduction}
\markboth{Introduction}{}
\addstarredchapter{Introduction}

\begin{flushright}
\emph{We shall not cease from exploration\\
And the end of all our exploring\\
Will be to arrive where we started\\
And know the place for the first time.\\} 
T.S. Eliot
\end{flushright}

The ``Zeptospace Odyssey'' by G. Giudice begins with this quote. It is an invitation to explore and understand the universe and, even in case of no discovery, the journey will expand the knowledge compared to where it started. Explorations at the Large Hadron Collider (LHC) started with the discovery of a Higgs boson with a mass of $\sim$125 \gev, announced by the ATLAS and CMS Collaborations, making the journey even more exciting and with new questions coming up.
Is this Higgs boson really the elementary scalar predicted by the Standard Model? Are there any hints of new physics in its properties? Therefore, the attention quickly shifted towards measuring its properties in order to determine whether it is the Standard Model Higgs boson or whether it has a completely different (e.g. composite) nature. Of particular interested are the Yukawa couplings, which in the Standard Model can be inferred from the measured fermion masses. At the LHC only the Yukawa couplings to third-generation fermions can be measured directly because of their high values. With the largest Yukawa coupling, $y_{t}\sim1$, the top quark may play a central role in the underlying electroweak symmetry breaking dynamics and any deviation of its Yukawa coupling can point to the presence of new physics. This is the motivation to focus, in this dissertation, on processes involving top quarks and Higgs bosons.\par
Firstly, the observation of the $t\bar{t}H$ process, from which the top Yukawa coupling can be extracted directly, is a stepping stone towards answering the key question about the nature of the Higgs boson. But with the discovery of the Higgs boson, is the Standard Model the end of the story in particle physics? Or is it just the start of a wonderful journey? Assuming the Standard Model as the final theory valid up to the Planck scale, it is not possible to explain the presence of a Higgs boson at the electroweak scale ($\sim$100 \gev) without invoking a fine tuning over 30 orders of magnitude, making the theory highly unnatural (\emph{naturalness problem}). Furthermore, the Standard Model does not include gravity and it explains only $\sim5\%$ of the energy density of the universe, lacking a candidate to explain the six times larger contribution from non-luminous matter (dark matter) present in the universe. Those are just few arguments on the fact that the Standard Model cannot be the final theory, but rather a successful low-energy description of a more complete theory at higher energy scales. Theories ``Beyond the Standard Model'' try to overcome some of these shortcomings,  e.g. by postulating new particles that can solve the problem of the unnatural mass of the Higgs boson and, in some cases, provide as well a dark-matter candidate. Some of the proposed solutions introduce extra spatial dimensions, compositeness or new strong sectors leading to the introduction of vector-like top-quark partners, which address the naturalness problem. Such particles can decay, through flavour-changing neutral-current interactions, in a top quark and a Higgs boson. These models predict as well an enhancement in the production of four-top-quark production, a rare process in the Standard Model.\par
This dissertation presents searches for $t\bar{t}H$ production, with $H\to b\bar{b}$, as well as new phenomena involving multiple top quarks and/or Higgs bosons, covering several extensions of the Standard Model discussed above. These searches use early Run 2 data collected with the ATLAS detector and share a final state characterised by the presence of high jet and $b$-jet multiplicities. The main background is $t\bar{t}+\ge1b$ production, which is very challenging to model. Sophisticated analyses are developed to maximize the sensitivity of the searches, through a combination of powerful discriminating variables between signal and background, and the determination of the background with high precision by exploiting high-statistics subsidiary data samples.
\par The content of this dissertation is organised as follows. Chapter \ref{chp:the} gives an introduction to the Standard Model, its shortcomings and proposed solutions, with particular attention to the BSM theories predicting new phenomena involving top quarks and Higgs bosons. The experimental setup used to produce and collect the data for these analyses are presented in Chapter \ref{chp:det}. Chapter \ref{chp:evtsim} describes the Monte Carlo techniques used to obtain simulated samples to compare with the data. Starting from the computation of the matrix element for a particular physics process,  Monte Carlo tools try to provide a complete picture for how. The validation of these tools is of crucial importance in the ATLAS Collaboration and the author was an active member of the team involved in this task. Chapter \ref{chp:obj} outlines the reconstruction and identification of physics objects, such as charged leptons, jets and weakly-interacting particles. Chapter \ref{chp:data} summarises the details of the collider data and simulated data samples used in this dissertation, as well as some useful tools for the analyses presented. Finally, the main topic of this dissertation, the search for new phenomena involving top quarks and Higgs bosons is discussed in Chapter \ref{chp:VLQ} and \ref{chp:ttH}, where two different analyses are presented.\par
The results presented in this dissertation have led to the following publications, in which the author has been one of the main analysers:
\bi
\ib ATLAS Collaboration, \emph{Search for production of vector-like top quark pairs and of four top quarks in the lepton-plus-jets final state in pp collisions at $\sqrt{s} =13$ TeV with the ATLAS detector}, \coloredLink{https://atlas.web.cern.ch/Atlas/GROUPS/PHYSICS/CONFNOTES/ATLAS-CONF-2016-013/}{ATLAS-CONF-2016-013} (2016).
\ib ATLAS Collaboration, \emph{Search for the associated production of a Higgs Boson with a top quark pair decaying into $H\to b\bar{b}$ pairs with the ATLAS detector}, \coloredLink{https://atlas.web.cern.ch/Atlas/GROUPS/PHYSICS/CONFNOTES/ATLAS-CONF-2016-080/}{ATLAS-CONF-2016-080} (2016).
\ib ATLAS Collaboration, \emph{Search for new phenomena in $t\bar{t}$ final states with additional heavy-flavour jets in pp collisions at $\sqrt{s}=13$ TeV with the ATLAS detector}, \coloredLink{https://atlas.web.cern.ch/Atlas/GROUPS/PHYSICS/CONFNOTES/ATLAS-CONF-2016-104/}{ATLAS-CONF-2016-104} (2016). 
\ei

\noindent Being involved in detailed studies of the $t\bar{t}+\ge1b$ background modelling, a key aspect for these analyses, the author has also contributed significantly to the following publications:

\bi
\ib ATLAS Collaboration, \emph{Studies of Monte Carlo generators in Higgs boson production for ATLAS Run 2}, \coloredLink{https://atlas.web.cern.ch/Atlas/GROUPS/PHYSICS/PUBNOTES/ATL-PHYS-PUB-2014-022}{ATLAS-PHYS-PUB-2014-022} (2014).
\ib ATLAS Collaboration, \emph{Additional studies of MC generator predictions for top quark production at the LHC}, \coloredLink{https://atlas.web.cern.ch/Atlas/GROUPS/PHYSICS/PUBNOTES/ATL-PHYS-PUB-2016-016/}{ATLAS-PHYS-PUB-2016-016} (2016).
\ib LHC Higgs Cross Section WG, \emph{Handbook of LHC Higgs Cross Sections: 4. Deciphering the Nature of the Higgs Sector}, \coloredLink{ https://arxiv.org/abs/1610.07922}{arXiv:1610.07922 [hep-ex]}.
\ei

\noindent The author was also the main analyser in a phenomenological study proposing a novel search for a light CP-odd scalar (see appendix \ref{App:AppendixA}): 

\bi
\ib M. Casolino \emph{et al.}, \emph{Probing a light CP-odd scalar in di-top associated production at the LHC}, \coloredLink{http://dx.doi.org/10.1140/epjc/s10052-015-3708-y}{Eur. Phys. J. C \textbf{75} (2015) 498}, \coloredLink{ http://arxiv.org/abs/1507.07004}{arXiv:1507.07004 [hep-ph]}.
\ei

\noindent Finally, the author has also contributed to the following results as part of the ATLAS Monte Carlo validation team:

\bi
\ib ATLAS Collaboration, \emph{Validation of Monte Carlo event generators in the ATLAS Collaboration for LHC Run 2}, \coloredLink{https://atlas.web.cern.ch/Atlas/GROUPS/PHYSICS/PUBNOTES/ATL-PHYS-PUB-2016-001/}{ATLAS-PHYS-PUB-2016-001} (2016).
\ei