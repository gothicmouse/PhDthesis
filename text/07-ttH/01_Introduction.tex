
After the discovery by the ATLAS and CMS Collaborations of a Higgs boson with mass of approximately of 125 \gev, the focus has shifted towards measurements of its properties in order to determine whether it is the SM Higgs boson or whether it has a completely different (e.g. composite) nature. Of particular importance is the top-Higgs Yukawa coupling which is close to unity, due to the large measured top-quark mass. Any deviation from its SM value might give insight on the scale of BSM physics. Indirect constraints are extracted through measurements of the Higgs-boson production rates at the LHC, since the top-Higgs Yukawa coupling yields the dominant contribution to the main production mode ($gg\to H$, or gluon fusion) and it also contributes to Higgs-boson decay to a photon pair ($H\to \gamma \gamma$). However, its contribution to these processes only enters through loop effects, which cannot be disentangled from other possible BSM contributions. The observation of the production of the Higgs boson in association with a pair of top quarks ($t\bar{t}H$) can measure directly the  magnitude of the top-Higgs Yukawa coupling, since at LO the cross section is proportional to the squared top-quark Yukawa coupling. Searches for $t\bar{t}H$ are performed in several final states according to the Higgs-boson decay mode. This chapter presents a search for $t\bar{t}H$ production with the Higgs boson decaying to a $b\bar{b}$ pair, which for a Higgs boson with $M_H=125$ $\gev$ is the dominant decay mode with a branching ratio of $58\%$. 


