
\label{chp:evtsim:intro}

A key point to test the SM or any possible extension of it is to quantify the consistency of the observed data with the theoretical prediction. 
An accurate simulation, including  the state-of-the-art understanding of the \emph{pp} collision physics and the experimental setup, is necessary to model physics processes and kinematic distributions.
Event simulation is a complex procedure divided into two major steps:
\bi
\ib Event generation: Simulation of the physics process of interest including the modelling of the partonic structure of the incoming protons, their collision and the subsequent event development up to the decays into stable particles. As the physics processes occurring in $pp$ collisions are probabilistic in nature, event generation involves pseudo-random numbers, which are generated using Monte Carlo (MC) techniques.
\ib Detector simulation and digitisation: MC techniques are used to simulate the geometry of the detector, the interaction of particles with the detector materials, and the corresponding detector response, including the digitisation of the detector electronics.
\ei

This chapter presents an overview of the simulation of $pp$ collisions, followed by a description of the MC generators used for the analyses in this dissertation, and the ATLAS detector simulation. It also includes a summary of the procedure followed in ATLAS to validate MC samples, which the author has participated in.\\