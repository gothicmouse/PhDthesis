The current understanding of the fundamental structure of matter and its interactions is summarised in the Standard Model of particle physics. This theory, developed in the 1960s, has turned out to be extremely valid in describing a large variety of observed phenomena and in predicting new ones that have found experimental confirmation. The discovery of the Higgs boson at the LHC in July 2012 was only the most recent proof of a tremendous successfully theory. The Standard Model is a very elegant and successful theory but there are several indications that cannot be the final theory of Nature. This chapter introduces the main ingredients of the Standard Model, describes its experimental tests, and provides a discussion of some of its limitations as the ultimate theory. Finally, an overview of some of the most promising theories beyond the Standard Model is provided.