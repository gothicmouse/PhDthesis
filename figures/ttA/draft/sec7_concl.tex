
%%%%%%%%%%%%%%%%%%%%%%%%%%%%%%%
\section{Conclusions}
\label{sec:conclusion}
%%%%%%%%%%%%%%%%%%%%%%%%%%%%%%%%

Searches for CP-odd scalars, as predicted by many extensions of the Standard Model and motivated by some recent astroparticle observations, are part of the core program of upcoming LHC runs at $\sqrt{s}=13$ and 14 TeV. Searches at LEP and during Run 1 of the LHC at $\sqrt{s}=7$ and 8 TeV have placed only weak constraints on the coupling strengths of CP-odd scalars with top and bottom quarks, or in their allowed mass range.

Using a simplified model approach for the signal, we have carried out a detailed study to evaluate the prospects at the LHC for probing scenarios with a 
CP-odd scalar with mass $20 \leq m_A < 100$ GeV, via the process $pp \to t\bar{t}A$ with subsequent decay $A\to b\bar{b}$. To separate the signal from the large background from $t\bar{t}$+jets production, we apply jet substructure techniques, reconstructing the mass of the CP-odd scalar as the mass of a 
large-radius jet containing two $b$-tagged subjets. 
The chosen method allows for a so-called 'bump hunt' over a fairly smooth background, and it may be the most promising strategy for searching for a CP-odd scalar with mass $\lesssim 50$ GeV, i.e. about twice the typical minimum $\pt$ cut for narrow jets used in standard LHC searches. A significant effort has been made in developing a semi-realistic experimental analysis, including a fairly complete description of systematic uncertainties and the usage of sophisticated statistical tools to constrain  {\em in-situ} the effect of systematic uncertainties, thus limiting their impact on the search sensitivity. We then derive expected upper limits on the production cross section times branching ratio using the CL$_{\rm{s}}$ method.

In specific models, e.g. 2HDM or NMSSM, the coupling of the $A$ boson with the top quark is related to other couplings in a well-defined way. Hence, the upper 
limits obtained on this coupling for a given mass $m_A$, can be used to bound other couplings of these models indirectly or as input for 
a global coupling fit. We find that in a type-I and type-II 2HDM the LHC can constrain a large fraction of the $(m_A, \tan \beta)$ parameter space, including 
the region preferred to explain the diffuse gamma-ray excess from the Galactic Centre as dark-matter annihilation via a CP-odd scalar mediator and decaying
into $b\bar{b}$. 
However, in the case of the NMSSM with a light CP-odd scalar, a Goldstone boson of either a spontaneously-broken R- or PQ-symmetry, 
the LHC appears to have very limited sensitivity in probing these models. 

Hence, depending on the concrete embedding of the scalar sector into a UV-complete theory, the LHC can provide complementary information, not accessible at either indirect detection experiments or electron-positron colliders, on the existence of CP-odd scalars, their mass and couplings to third-generation fermions.