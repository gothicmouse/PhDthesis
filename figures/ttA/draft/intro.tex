%%%%%%%%%%%%%%%%%%%%%%%%%%%%%%%%%%%%%%%%%%%%%%%%%%%
\section{Introduction}
\label{sec:intro}
%%%%%%%%%%%%%%%%%%%%%%%%%%%%%%%%%%%%%%%%%%%%%%%%%%%%%
%%%%%%%%%%%%%%%%%%%%%%%%%%%%%%%%%%%%%%%%%%%%%%%%%%%%%
The recent discovery of the Higgs boson \cite{Chatrchyan:2012ufa,Aad:2012tfa} marked a new era for fundamental physics. For the first time an electroweak-scale scalar resonance has been discovered, supposedly a remnant of the mechanism underlying electroweak symmetry breaking \cite{orig}.

While elementary scalar particles have been observed in nature for the first time, they are often an integral part of Standard Model (SM) extensions, e.g. Supersymmetry or general N-Higgs Doublet Models. When these extensions contain complex scalar fields, as a result, CP-odd scalars are introduced to the spectrum of the theory. Hence, since their existence would be evidence for physics beyond the SM, searches for CP-odd scalars are at the core of the 
current LHC program. 

Recently, CP-odd scalars as mediators between Dark Matter (DM) and SM particles have received attention in explaining the diffuse gamma-ray excess from the Galactic Centre~\cite{Goodenough:2009gk,Hooper:2010mq,Abazajian:2012pn,Daylan:2014rsa} in the contexts of the so-called Coy Dark Matter models~\cite{Boehm:2014hva,Hektor:2014kga,Arina:2014yna}, and the Next-to-Minimal-Supersymmetric-Standard-Model (NMSSM)~~\cite{Cheung:2014lqa,Huang:2014cla}. Hence, they are included as mediators in simplified models by the ATLAS and CMS collaborations to recast searches for  jets and missing transverse energy (`monojet') during the upcoming LHC runs \cite{Malik:2014ggr,Abdallah:2014hon}.

In contrast to the widely accepted paradigm that new physics particles have to be heavy, i.e. masses of $\mathcal{O}$(1) TeV or beyond, the mass of CP-odd scalars is almost unconstrained by direct searches. As interactions between gauge bosons and CP-odd scalars are only induced via higher-dimensional operators, e.g. $\frac{1}{2} A \epsilon_{\mu \nu \sigma \rho} V_{\sigma \rho} V^{\mu \nu}$, limits from LEP are fairly weak. The main collider sensitivities may mainly arise from bottom quark or top quark-associated productions (for recent explorations, see~\cite{Craig:2015jba,Hajer:2015gka}). Further, due to the predicted velocity suppression in direct detection experiments for CP-odd scalar mediators, even light CP-odd scalars are still in agreement with experimental observations. Some constraints from flavour physics exist but limits are again weak if $m_A \gtrsim 5$ GeV \cite{Dolan:2014ska}, assuming the CP-odd scalar interacts with fermions in agreement with the hypothesis of minimal flavour violation \cite{D'Ambrosio:2002ex}. 

Therefore, indirect detection experiments and direct searches at the LHC appear to be the most sensitive ways to search for the existence of electroweak-scale CP-odd scalar particles. In this paper, instead of previously explored paths of searching for CP-odd scalars in gluon-fusion production \cite{Klamke:2007cu,Dolan:2014upa}, we focus on the direct production of such particles in association with a top quark pair and subsequent decay into a bottom quark pair, $p p \to t \bar{t} A \to t \bar{t} b \bar{b}$. Thus, we derive limits on the mass and coupling strength of the CP-odd scalar in a process with unsuppressed fermion couplings only.\footnote{We note that, if the pseudoscalar couples for example in a universal way to fermions as part of a UV-complete model, thereby not respecting Yukawa-like coupling hierarchies, other production and decay channels might be more sensitive. However, the analysis we provide is still valid as a subset of possible search channels.}

This paper is organised as follows. In Sec.~\ref{sec:model} we briefly outline the way we incorporate the CP-odd scalar into the theory, using a simplified model approach. The event generation and details of the final state reconstruction are described in detail in Secs.~\ref{sec:generation} and \ref{sec:analysis}. In Sec.~\ref{sec:limit} we derive limits on the mass of the CP-odd scalar and its couplings to top quarks. Such limits can be applied to models where the CP-odd scalar arises as part of a Higgs multiplet. We recast these limits in the context of the 2HDM and the NMSSM in Sec.\ref{sec:interpretation}. Finally, in Sec.\ref{sec:conclusion} we offer conclusions.

